\section{Preliminaries}\label{sec:preliminaries}

\subsection{Boolean Networks and Fixed Points}

A Boolean Network (BN) \(\bn\) is defined as a finite set of Boolean functions over a finite set of Boolean variables, denoted by \(\var{\bn}\).
Each variable \(v \in \var{\bn}\) is associated with a Boolean function \(f_v \colon \twod^{|\var{\bn}|} \to \twod\).
A function \(f_v\) is termed \emph{constant} if it is always either 0 or 1 regardless of the values of its arguments.
A variable \(v\) is considered a \emph{source variable} if \(f_v\) is the identity function on \(v\), i.e., \(f_v = v\).
A state \(s\) of \(f\) is a Boolean vector \(s \in \twod^{|\var{\bn}|}\) that can be viewed as a mapping: \(s \colon \var{\bn} \to \twod\);
we denote the value of variable $v$ in state $s$ by \(s_v\).
For convenience, a state is often represented as a string of values (e.g., ``0110'' instead of (0, 1, 1, 0)).

At each discrete time step \(t\), each variable \(v\) can update its state  according to its Boolean function $f_v$; that is, $v$'s state at time $t+1$ is given by \(s'_v = f_v(s)\).
An \emph{update scheme} specifies how these state updates occur over time~\cite{SKIKK2020}.
The two primary schemes are synchronous, in which all variables update simultaneously, and fully asynchronous, where a single variable is chosen non-deterministically to update.
Under arbitrary update scheme, the BN transitions from one state to another --- a process known as a {\em state transition}. The overall dynamics of the BN are captured by the {\em State Transition Graph} (STG), a directed graph whose nodes represent states and edges represent transitions.
We denote the STG under the synchronous update scheme as  \(\stg{\bn}\) and that under the fully asynchronous scheme as \(\atg{\bn}\).

A non-empty set \(A\) of states is a \emph{trap set} if there is no transition from a state in $A$ to a state outside $A$ in the State Transition Graph (STG) of $\bn$ (i.e., there is no pair $x \in A$ and $y \not \in A$ such that \((x, y)\) is an arc in the STG)~\cite{HAH2015}.
A trap set that is minimal with respect to set inclusion is termed an {\em attractor}.
In particular, an attractor containing a single state is called a {\em fixed point}, while one with two or more states is referred to as a {\em cyclic} attractor.
A \emph{sub-space} \(m\) of a BN \(\bn\) is a mapping \(m \colon \var{\bn} \to \threed\).
A variable \(v \in \var{\bn}\) is said to be \emph{fixed} (resp.\ \emph{free}) in \(m\) if \(m(v) \neq \star\) (resp.\ \(m(v) = \star\)).
For convenience, a sub-space is often represented as a string of values  (e.g., \(0\star\) instead of \(\{v_1 = 0, v_2 = \star\}\)).
The sub-space \(m\) represents a set of states, denoted by \(\mathcal{S}[m]\), defined as 
\[
\mathcal{S}[m] = \{s \in \mathbb{B}^{|\var{\bn}|} \mid s_v = m(v), \forall v \in \var{\bn}, m(v) \neq \star\}
\]
For example, if \(m = \star11\), then \(\mathcal{S}[m] = \{011, 111\}\).
If a sub-space is also a trap set, it is a \emph{trap space}.
Unlike trap sets and attractors, trap spaces are independent of the update scheme employed~\cite{HAH2015}.
Notably, a fixed point of \(\bn\) is a special trap space in which all variables are fixed.
A trap space \(m\) is \emph{minimal} if there is no trap space \(m'\) such that \(\mathcal{S}[m'] \subset \mathcal{S}[m]\).
Since an attractor is a subset-minimal trap set, a minimal trap space contains at least one attractor of the BN, regardless of the update scheme employed~\cite{HAH2015}.

\subsection{Fixed-Point Counting Problem Variants}

\paragraph{C-FIX-1:}

Test

\paragraph{C-FIX-2:}

Test

\paragraph{C-FIX-3:}

Test

\subsection{Related Work on Fixed-Point Counting}

\subsection{Abstract Dialectical Frameworks}

\subsection{Network Reduction Techniques}

\paragraph{Stripping Output Nodes:}

Test

\paragraph{Propagating Fixed Nodes:}

Test

\paragraph{Suppressing Non-Autoregulated Nodes:}

Test