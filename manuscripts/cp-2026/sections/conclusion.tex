\section{Conclusion}\label{sec:conclusion}

This paper addresses the problem of counting minimal trap spaces and fixed points in Boolean networks.
These are critical concepts in understanding of long-term BN behavior and are relevant across a diverse set of application domains, including systems biology, abstract argumentation, and logic programming. Trap space counting is especially important in systems biology: due to the inherent robustness of biological phenomena, biologically motivated BNs admit a high degree of redundancy, resulting in a vast repertoire of closely related trap spaces (or fixed points) that cannot be explored solely through enumeration.

Here, we propose novel methods for determining trap space and fixed point counts using approximate answer set counting, thus entirely avoiding costly enumeration. We apply this methodology to three biologically motivated problems: (a) general counting; (b) counting occurrences of a known biological phenotype, and (c) projected counting of perturbations that ensure the emergence of a known biological phenotype. The last problem is particularly timely, as it allows us to determine \emph{perturbation robustness}~\cite{BBPSS2023,kitano2007towards}, a vital measure that determines how stable a phenotype appears under external stimuli. Through extensive experiments on a diverse set of benchmarks, we show that approximate counting substantially improves the feasibility of counting in this domain, outperforming traditional enumeration-based and exact approaches whenever applicable.

Our work opens several promising directions for future research:
First is to integrate reduction techniques---previously shown to be effective in BN analysis~\cite{Rozum2021,ST23,TP2024}---to further improve counting accuracy and scalability. 
Second direction explores hybrid strategies that combine exact and approximate counting, aiming to strike a balance between efficiency and precision. 
Finally, a deeper investigation into the computational complexity of the counting problems would help refine our understanding of their theoretical underpinnings.