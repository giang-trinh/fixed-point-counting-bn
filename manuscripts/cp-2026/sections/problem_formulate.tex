\section{Problem Formulation}\label{sec:problem-formulation}

In this section, we introduce several counting problems related to minimal trap spaces in Boolean networks (BNs), covering both minimal trap spaces and fixed points.
% These problems encompass both minimal trap spaces and fixed points.
We begin with a straightforward counting variant, 
then propose a specialized variant that requires solutions to satisfy a specific {\em property}, 
and finally present a more complex variant focused on phenotype measurement in BNs under perturbations. To highlight the biological utility of these theoretical problems, a brief case study is also given in Appendix~\ref{sec:case-study}.

% As mentioned early in Section~\ref{sec:introduction}, several counting problems w.r.t. fixed points in BNs have been studied.
% A fixed point is a special minimal trap space where all variables are fixed.
% Hence, we hereafter formulate three types of counting problems w.r.t. minimal trap spaces in BNs, and each type is divided into two versions, one for fixed points only and one for (general) minimal trap spaces.

\subsection{Counting Minimal Trap Spaces and Fixed Points}\label{subsec:formulation-first}

We introduce two fundamental counting problems for Boolean networks (BNs): one that counts the number of minimal trap spaces (\Cref{def:problem-first-MTS}) and another one counts the number of fixed points (\Cref{def:problem-first-FIX}). 
These problems address the basic question: How many minimal trap spaces or fixed points does a given BN have?

\begin{definition}[\acfirstmts{}]\label{def:problem-first-MTS}
	Given a BN \(\bn\), compute the number of minimal trap spaces of \(\bn\).
\end{definition}

\begin{definition}[\acfirstfix{}]\label{def:problem-first-FIX}
	Given a BN $\bn$, compute the number of fixed points of $\bn$.
\end{definition}

% Next, we discuss in detail the implications of these two problems in three domains, including BNs, abstract argumentation, and logic programming.

In BN research, both counting problems --- \acfirstmts{} and \acfirstfix{} --- are valuable when full enumeration is infeasible, as in gene regulatory network models with many source variables~\cite{SVATSAJ2020,TBPS2024,TBS2023}.
They are also useful when divergent solutions are sought~\cite{SVLAL2020}. 
Notably, the fixed point counting problem (\acfirstfix{}) can enhance methods for enumerating asynchronous attractors by enabling the selection of a smaller candidate set, thereby potentially speeding up the filtering process~\cite{TKB2022}.
Finally, both \acfirstmts{} and \acfirstfix{} can lay the groundwork for probabilistic reasoning in BNs~\cite{SDZ2002}.

Recent research has connected Boolean networks (BNs) with two broad fields: abstract argumentation and logic programming.
Although we omit the preliminaries on Abstract Argumentation Frameworks (AFs)~\cite{DFGH2022}, Abstract Dialectical Frameworks (ADFs)~\cite{LMNWW22}, and Normal Logic Programs (NLPs)~\cite{TBSF2024}, interested readers can consult the cited papers for details.
Specifically, Trinh et al.~\cite{TBV2025} and Dimopoulos et al.~\cite{DWM2024}  demonstrate that the {\em preferred} (resp.\ {\em stable}) extensions of an AF one-to-one correspond to the minimal trap spaces (resp.\ fixed points) of the respective BN.
Similarly,  Heyninck et al. \cite{HMJ2024} and Azpeitia et al.~\cite{ASDO2024} demonstrate that ADFs and BNs are identical --- in this context, the {\em preferred} (resp.\ {\em two-valued}) interpretations of an ADF match the minimal trap spaces (resp.\ fixed points) of the respective BN.
In the realm of logic programming, the subset-minimal {\em supported partial} models (resp.\ {\em supported} models) of a {\em finite ground} NLP correspond one-to-one with the minimal trap spaces (resp.\ fixed points) of the respective BN~\cite{Inoue2011,TBSF2024}.
Moreover, if the finite ground NLP is {\em tight}~\cite{Fages1994,LL03}, then its regular models (resp.\ stable models) one-to-one correspond to the minimal trap spaces (resp.\ fixed points) of the respective BN.
These types of extensions, interpretations, and models are central to the study of AFs, ADFs, and finite ground NLPs~\cite{FHM2024,JNSSY2006,LMNWW22}.

Recent work on AFs has focused on counting stable and preferred extensions, yielding both complexity results~\cite{FHM2024} and dynamic programming methods~\cite{DFGH2022} (note that the dynamic programming methods do not support preferred extensions).
In contrast, counting problems for ADFs and finite ground NLPs remain largely unexplored, aside from a few studies on answer sets~\cite{DBLP:conf/aaai/AzizCMS15,KESHFM22}.
Thanks to the connections between BNs and these formalisms, these results from \acfirstmts{} and \acfirstfix{} can be extended to ADFs --- where they correspond to preferred and two-valued interpretations --- as well as to general finite ground NLPs (for supported partial and supported models) and tight finite ground NLPs (for regular and stable models).
% Notably, our method for \acfirstmts{} (ref.~\ref{subsec:methods-first}) is the first approach for counting preferred interpretations in ADFs and regular models in tight finite ground NLPs.

\subsection{Counting with Satisfying Properties}\label{subsec:formulation-second}

We examine a specialized variant of problems \acfirstmts{} and \acfirstfix{}, focusing on counting minimal trap spaces (Definition~\ref{def:problem-second-MTS}) and fixed points (Definition~\ref{def:problem-second-FIX}) that satisfy a specified property. 
This formulation addresses the natural question: How many minimal trap spaces (or fixed points) in a BN exhibit a given property or assumption?

In systems biology, BNs are used to model biological phenotypes, which reflect an organism's functional characteristics~\cite{LQADZXHE2016}.
Several definitions of phenotype in BNs have been proposed~\cite{BBPSS2023,KHNS2018,KTFJCS2021}; in this work, we adopt one of the most widely used notions.
%To propose the second version of counting problem, we introduce the notion of {\em phenotype} of a BN~\cite{KTFJCS2021,KHNS2018}.
Given a BN $f$, we define {\em trait} as a statement of the form $(v \leftrightarrow e)$, where $v \in \var{f}$ and $e \in \threed$.
Note that \(v \leftrightarrow \star\) is evaluated true if and ony if \(v = \star\).
A phenotype $\phen$ is then defined as the conjunction of a set of traits.
A sub-space $m$ satisfies a phenotype $\phen$ (denoted by \(m \models \phen\)) if, upon replacing each variable $v \in \phen$ with its value $m(v)$, the resulting formula evaluates to true under propositional semantics.
Unlike minimal trap spaces, fixed points require that all variables take Boolean values ($e \in \twod$), and hence phenotypes involving `$\star$' are not meaningful in this context.

In our problem formulation, the property of interest is a desirable phenotype.
In systems biology, a minimal trap space satisfying a phenotype suggests the phenotype's potential emergence in vivo.
% or a hypothesis that is changed and tested frequently during the modeling cycle~\cite{SVATSAJ2020}.
Furthermore, this counting variant is applicable beyond systems biology. 
For instance, in abstract argumentation frameworks (resp.\ normal logic programs), a phenotype may represent the presence of particular set of arguments (resp.\ atoms) in an extension (resp.\ a model)~\cite{FHM2024,JNSSY2006}, a concept closely linked to {\em credulous reasoning}.

%To define the second variant of counting problems, we first introduce the phenotype of a BN.
%A phenotype $\phen$ of a BN \(\bn\) is a conjunction of {\em traits}, where each trait is represented by the {\em activity} (or {\em inactivity}) of a particular variable in BN.
%For example, \((v_1 = 1)\) is a trait, and \((v_1 = 1 \land v_3 = 0)\) is a phenotype consisting of two traits.
%For convenience, we consider a phenotype as a sub-space where variables not present in phenotype are free.
%A sub-space \(m\) of \(\bn\) \emph{satisfies} the phenotype \(\phen\) (denoted by \(m \models \phen\)) iff \(m(v) = \phen(v)\), for each variable \(v \in \var{\bn}\) such that \(\phen(v) \neq \star\).

% To define the second variant of counting problems, we first introduce the notion of \emph{property} in BNs.
% Given a BN $\bn$, we define a \emph{property} \(\prop\) as a propositional formula consisting of atomic terms of the form \(v = b\), where \(v \in \var{\bn}\) and \(b \in \threed\).
% A sub-space \(m\) satisfies a property \(\prop\) if replacing every variable \(v \in \prop\) by \(m(v)\) evaluates \(\prop\) to true.

\begin{definition}[\acsecondmts{}]\label{def:problem-second-MTS}
	Given a BN \(\bn\) and a phenotype $\phen$, compute the number of minimal trap spaces of \(\bn\) that satisfy $\phen$.
\end{definition}

\begin{definition}[\acsecondfix{}]\label{def:problem-second-FIX}
	Given a BN \(\bn\) and a phenotype $\phen$, compute the number of fixed points of \(\bn\) that satisfy $\phen$.
\end{definition}


When the specified phenotype is a {\em tautology}, \acsecondmts{} (resp. \acsecondfix{}) reduces directly to \acfirstmts{} (resp. \acfirstfix{}).
Thus, all implications discussed for \acfirstmts{} and \acfirstfix{} also apply to \acsecondmts{} and \acsecondfix{}, respectively.
In the following, we examine these applications through the lens of abstract argumentation frameworks.

First, \acsecondmts{} and \acsecondfix{} facilitate more nuanced reasoning between {\em skeptical} and {\em credulous} approaches by integrating quantitative and probabilistic methods~\cite{FHN2022}. 
For instance, by running \acsecondmts{} (or \acsecondfix{}) twice --- once with a specific set of arguments and once with a tautology --- we can gain insights into preferred (or stable) extensions~\cite{FHM2024}.
Second, this approach shifts reasoning from simple decision-making (i.e., determining whether a set of arguments is present in an extension) to probabilistic reasoning (i.e., assessing the likelihood of a set of arguments appears in an extension). 
Moreover, this framework supports the development of advanced probabilistic semantics for abstract argumentation frameworks~\cite{KBDDGH2022}.
To our knowledge, these aspects have not yet been explored in the context of BNs, and our work paves the way for integrating such reasoning into BN analysis.

\begin{example}\label{exam:four-first-problems}
	Consider again the BN \(\bn\) shown in~\Cref{exam:straight-BN}.
	It has a unique minimal trap space \(m_1 = 00\), which is also its only fixed point of \(\bn\).
	Hence, the answer to \acfirstmts{} (resp.\ \acfirstfix{}) is $1$.
	Considering the phenotype \(\phen = (b \leftrightarrow \star)\), the answer to \acsecondmts{} is $0$.
	%For another phenotype \(\prop \equiv ((a = 0) \lor (b = \star))\), the answer to both $\acsecondfix{}$ and $\acsecondmts{}$ is $1$.
\end{example}

\subsection{Counting Under Perturbations and Measuring Robustness}\label{subsec:formulation-third}
%We define a third variant of the counting problems as a specialized case of \acsecondmts{} and \acsecondfix{}.
We consider a main contribution of this paper to be the identification of new, relevant counting problems (other than \acsecondmts{} and \acsecondfix{}, which were previously known): \acthirdmts{} and \acthirdfix{}.
To formalize these problems, we introduce the concept of a {\em perturbation}.

Consider a BN \(\bn\).
A \emph{perturbation} $\sigma$~\cite{su2020sequential} on a set \(\pert \subseteq \var{f}\) of {\em perturbable} variables is defined as a mapping from \(\pert \to \mathbb{B}_{\star}\).
In practice, $\pert$ can be any subset of variables whose perturbation is biologically meaningful.
Since each variable in $\pert$ can assume one of three values under perturbation, there are \(3^{|\pert|}\) possible perturbations.
For each \(v \in \pert\), setting \(\sigma(v) = 0\) forces \(f_v = 0\) ({\em knockout perturbation}), setting \(\sigma(v) = 1\) forces \(f_v = 1\) ({\em over-expression perturbation}), and setting \(\sigma(v) = \star\) leaves \(f_v\) unchanged.
Consequently, the perturbed BN, denoted \(\bn^{\sigma}\), is defined by \(\var{\bn^{\sigma}} = \var{\bn}\) and, for every \(v \in \var{\bn^{\sigma}}\), 

$$
f^{\sigma}_v =\begin{cases}
		 \sigma(v) & \text{if \(v \in \pert\) and \(\sigma(v) \neq \star\)}\\
         f_v & \text{otherwise}
		 \end{cases}
$$Note that, the value \(\star\) is used to distinguish imperturbable variables and perturbable variables that are unchanged under a certain perturbation.

In biological systems, a perturbation refers to any disturbance that disrupts the normal functioning of a BN. 
Such disturbances may arise from {\em genetic mutations}~\cite{shmulevich2002gene}, external factors such as {\em medications}~\cite{bloomingdale2018boolean}, or other influences~\cite{montagud2022patient}. 
These perturbations can substantially alter the phenotypes exhibited by a BN, making it crucial to quantify their effects on network behavior. 
In systems biology, this impact is commonly assessed in terms of {\em robustness}~\cite{kitano2007towards}, which motivates our definitions of problems \acthirdmts{} and \acthirdfix{}.
%Inspired by the term \emph{phenotype} in systems biology, we introduce the notion of \emph{phenotype} in BNs.
%A phenotype $\phen$ of a BN \(\bn\) is a conjunction of {\em traits}, where each trait is represented by the {\em activity} (or {\em inactivity}) of a particular variable in the BN.
%For example, \((v_1 = 1)\) is a trait, and \((v_1 = 1 \land v_3 = 0)\) is a phenotype consisting of two traits.
%It is easy to see that a phenotype is a special property.

\begin{definition}[\acthirdmts{}]\label{def:problem-third-MTS}
	Given a BN \(\bn\), a set of perturbable variables \(\pert\), and a target phenotype \(\phen\),
	determine the number of perturbations \(\sigma\) on \(\pert\) such that the perturbed BN \(\bn^{\sigma}\) exhibits at least one minimal trap space that satisfies \(\phen\).
\end{definition}

\begin{definition}[\acthirdfix{}]\label{def:problem-third-FIX}
	Given a BN \(\bn\), a set of perturbable variables \(\pert\), and a target phenotype \(\phen\),
	determine the number of perturbations \(\sigma\) on \(\pert\) such that the perturbed BN \(\bn^{\sigma}\) exhibits at least one fixed point that satisfies \(\phen\).
\end{definition}

% Here, phenotype $\phen$ is given as a trap space property (as defined for \acsecondfix{} and \acsecondmts{}). 
Leveraging the results of \acthirdmts{} (or \acthirdfix{} if focusing solely on fixed points), we define the robustness of a phenotype as the fraction of perturbations that preserve the phenotype relative to the total number of perturbations ($3^{\Card{\pert}}$). 
In other words, robustness measures the probability that a phenotype remains active following a random, admissible perturbation is applied to the network. 
This measure of phenotype robustness can inform the selection of specific perturbations and guide targeted treatment strategies~\cite{BBPSS2023,TP2024}. A small case study on an interferon model with 121 variables presents these concepts more practically in Appendix~\ref{sec:case-study}.

%{\color{red}
%	\begin{itemize}
%		\item PRM-MTS (phenotype robustness measurement on minimal trap spaces)
%		\item PRM-FIX (phenotype robustness measurement on fixed points)
%	\end{itemize}
%}

\begin{example}\label{exam:third-MTS}
	Consider BN \(\bn\) given in~\Cref{exam:straight-BN}.
	Consider $\pert = \{b\}$ denote the set of perturbable variables and define the set of desirable phenotype as \(\phen = (a \leftrightarrow 0 \land b \leftrightarrow 0)\).
	There are three possible perturbations: \(\sigma_1 = \{b = 0\}\), \(\sigma_2 = \{b = 1\}\), and \(\sigma_3 = \{b = \star\}\).
	The perturbation \(\sigma_1\) represents the knockout perturbation of variable \(b\) and \(\stg{f^{\sigma_1}}\) is given in~\Cref{fig:sstg-straight-BN-wildtype-knockout}b.
	It is straightforward to observe that \(\bn^{\sigma_1}\) has two minimal trap spaces $00$ and $10$.
	In constrast, the BN \(\bn^{\sigma_2}\) has one minimal trap space $01$, which does not satisfy the given phenotype.
	The BN \(\bn^{\sigma_3}\) equals \(\bn\) and has one minimal trap space $00$.
	Therefore, for BN $f$, phenotype $\phen$ and perturbable variables $\pert$, the answer to $\acthirdmts{}$ is $2$ 
 	%Hence, \(\acthirdmts{}(f, \pert, \phen) = 2\).
	and the perturbation robustness of phenotype $\phen$ in BN \(\bn\) w.r.t. $\pert$ is \(\sfrac{2}{3}\).
\end{example}


\noindent\textbf{On the impact of precision} Finally, it should be noted that while these problems are defined \emph{exactly}, in practice their results mainly serve as a means of comparison. For example, the results of C-MTS-2 could be used to compare the abundance of two biological phenotypes. Then the exact count may not be important, as long as the two phenotypes can be compared reliably. As such, these problems are particularly suitable for approximate counting. This also impacts the choice of method parameters ($\epsilon$ and $\delta$), as in practice, even low precision (as used in our benchmarks) can be sufficient to distinguish between significantly different phenotypes. For closely matched results, the precision can be then increased as needed.

